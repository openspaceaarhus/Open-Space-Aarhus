\documentclass{beamer}

% \mode<presentation>
\usetheme{Luebeck} %Warsaw
% \usecolortheme{seahorse}
% \usecolortheme{lily}
\usecolortheme[RGB={51,255,0}
]{structure}

\usepackage[utf8]{inputenc}
\usepackage[danish]{babel}
\usepackage{graphicx}
\usepackage{url}

% \setbeameroption{show notes} %vis noterne på slides
% \setbeameroption{hide notes} %skjul noter
% \insertslideintonotes{0.30} %indsætter mini-slide på note-sider
% \setbeameroption{show only notes} %til udprintning af mine noter

% \setbeamercolor{alerted text}{fg=black} %alert-tekst-farve
\graphicspath{{img/}}
\title[${T^3}$ - Tech Talk Tuesday]{${T^3}$ - Tech Talk Tuesday \\
we start 19.05 }
\author{Open Space Aarhus}
\date{\today}
\institute[Bryggervej 30]{Bryggervej 30, 8240 Århus N}

\pgfdeclareimage[height=1.3cm]{university-logo}{osaa_logo_neon_rgb}
\logo{\pgfuseimage{university-logo}}

\begin{document}

\begin{frame}[label=titlepage]
  \titlepage
\end{frame}

\begin{frame}
  \frametitle{Foreningen}
  \begin{itemize}
  \item Idé - fælleslokaler til nørder og foreninger
  \item Bliv medlem:
    \begin{itemize}
    \item Adgang til lokalerne 24/7
    \item Det er det, foreningen er baseret på
    \item Gå ind på  \url{https://hal.osaa.dk/hal}
    \end{itemize}
  \item Hvem bruger lokalerne:
    \begin{itemize}
    \item Hack Århus - mødes hver tirsdag kl. 17-23
    \item Ubuntu Danmark
    \item Seniorer uden Grænser
    \item Din forening?
    \end{itemize}
  \item Vær med i OSAA:
    \begin{itemize}
    \item Brug lokalerne til dine aktiviteter
    \item Deltag i eksisterende aktiviteter - se \url{http://osaa.dk}
    \item Åbne bestyrelsesmøder hver torsdag efter ${T^3}$
    \end{itemize}
  \end{itemize}						
\end{frame}

\begin{frame}
  \frametitle{Tech Talk Tuesday}
  \begin{itemize}		  
  \item Hvad - små tekniske foredrag på 15-30 min.
  \item Hvorfor - få et indblik i mange forskellige emner
  \item Hvornår - Hver 1. tirsdag i måneden kl. 19
  \item Hvem - Kom til os, hvis du har ideer til emner eller oplægsholdere
  \item Hvor - \url{http://t3.osaa.dk}
  \end{itemize}						
\end{frame}

\begin{frame}
  \frametitle{Aftenens program}
  \begin{itemize}
    % \item Nyt fra OSAA - \emph{hvad sker der i lokalerne?}
  \item pedersen: \emph{OSAA} siden sidst (10 min)
  \item 5 min pause - snacks
  \item  (30 min) Mark Urup: Embedded Linux 
  \item 5 min pause
  \item  (30 min) Bertho: En ikke så ringe klokke
  \item Workshops

  \end{itemize}						
\end{frame}

\begin{frame}
  \frametitle{Hvad er der sket siden sidst?}
  \begin{itemize}

  \item Vi har fået \emph{INTERNETS}
  \item Workshop LT-spice
  \item Fået nye T-shirt designs \url{https://osaa.spreadshirt.dk}
  \item Rengøringsbounty
  \item Teslaklubben?   
  \item \emph{LASER projektet}  
  \end{itemize}						
\end{frame}


\begin{frame}
  \frametitle{Hvad er der sket siden sidst i ubuntu} Ubuntu - vi har
  startet et usability-test team, det første i verden, har lavet
  testcases og har også været igang med at udføre nogle af dem og
  optage det på video det vi finder ud af, sender
  vi videre til ubuntu-teamet, som så kan bruge det i deres arbejde
   vi er ved at planlægge et release-arrangement
  for den nyeste ubuntu 
 
hvis der er interesse, begynder vi nok snart at lave
instruktionsvideoer på dansk om brug og installation næste møde er på
torsdag d. 7. april kl. 17
				
\end{frame}



\begin{frame}
  \frametitle{Hvor kan jeg se hvad der sker?}
  \textbf{Følg med på}
  \begin{itemize}
  \item  \url{http://osaa.dk}
  \item  \url{https://groups.google.com/group/openspaceaarhus}
  \item  \url{http://hackaarhus.dk/forum}
  \item  \url{http://twitter.com} \texttt{\#osaa}
  \item  \url{http://facebook.com} 
  \end{itemize}						
\end{frame}


\begin{frame}
  \frametitle{Kommende aktiviteter}
  \begin{itemize}
  \item OSAA bestyrelsesmøde: Torsdag d. 3/3 2011 kl. 19	 			
  \item Næste weekend Open Source Days - \textbf{tinkertrack}
  \item Næste Næste weekend, \url{http://fo-workshop.osaa.dk}
  \item Teslaklubben?
  \item Hack Århus: Tirsdage kl. 17-23	       	
  \item Ubuntu Århus: Torsdage i lige uger kl. 17-19 
%  \item Python-gruppe: Følg med på \url{http://www.pygroups.dk/}
  \item Næste ${T^3}$: Tirsdag d. 1/3 2011 kl. 19.00
  \end{itemize}						
\end{frame}

\begin{frame}
  \frametitle{Pause}
  \begin{itemize}
  \item Snak med dine mednørder		
  \item Køb snacks 
  \item Hvis du har spørgsmål/ideer til OSAA, så find et af bestyrelsesmedlemmerne
  \end{itemize}	
  % \includegraphics[width=1.0\textwidth]{Neverending}					
\end{frame}




% \begin{frame}
%   \frametitle{Kommende arrangementer}
%   \begin{itemize}
%   \item Vi ses næste gang til:
%     \begin{itemize}
%     \item Hack Århus klubaften hver tirsdag kl. 17-23	       	
%     \item Ubuntu Århus torsdage i lige uger kl. 17-19 
%     \item OSAA bestyrelsesmøde torsdag d. 4/11 2010 kl. 19	 	
%       
%     \item Ubuntu release-arrangement lørdag d. 20. november	
%     \item Næste ${T^3}$ tirsdag d. 7/12 2010 kl. 19.00
%     \end{itemize}						
%   \item Følg med på \url{http://osaa.dk}
%   \end{itemize}						
% \end{frame}

% \begin{frame}
%   \frametitle{Lidt om foreningen}
%   \begin{itemize}
%   \item 
%			\item 
%   \end{itemize}						
% \end{frame}
% \begin{frame}
%   \frametitle{Oversigt}
%   \tableofcontents
% \end{frame}

\end{document}

