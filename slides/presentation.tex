\documentclass{beamer}

% \mode<presentation>
\usetheme{Luebeck} %Warsaw
% \usecolortheme{seahorse}
% \usecolortheme{lily}
\usecolortheme[RGB={51,255,0}
]{structure}

\usepackage[utf8]{inputenc}
\usepackage[danish]{babel}
\usepackage{graphicx}
\usepackage{url}

% \setbeameroption{show notes} %vis noterne på slides
% \setbeameroption{hide notes} %skjul noter
% \insertslideintonotes{0.30} %indsætter mini-slide på note-sider
% \setbeameroption{show only notes} %til udprintning af mine noter

% \setbeamercolor{alerted text}{fg=black} %alert-tekst-farve
\graphicspath{{img/}}
\title[${T^3}$ - Tech Talk Tuesday]{${T^3}$ - Tech Talk Tuesday \\
we start 19.05 }
\author{Open Space Aarhus}
\date{\today}
\institute[Katrinebjergvej 105]{Katrinebjergvej 105, 8200 Aarhus N}

\pgfdeclareimage[height=1.3cm]{university-logo}{osaa_logo_neon_rgb}
\logo{\pgfuseimage{university-logo}}

\begin{document}

\begin{frame}[label=titlepage]
  \titlepage
\end{frame}

\begin{frame}
  \frametitle{Foreningen}
  \begin{itemize}
    \item Idé - fælleslokaler til nørder og foreninger
    \item Bliv medlem:
    \begin{itemize}
      \item Adgang til lokalerne 24/7
      \item Det er det, foreningen er baseret på
      \item Gå ind på http://ny.osaa.dk
      \item Hvem bruger lokalerne:
      \begin{itemize}
        \item Hack tirsdag - Vi mødes hver tirsdag kl. ~16->sent
        \item TASF mødes om onsdagen og nørder arkadespil
        \item Din forening?
      \end{itemize}
      \item Vær med i OSAA:
      \begin{itemize}
        \item Brug lokalerne til dine aktiviteter
        \item Deltag i eksisterende aktiviteter - se \url{http://osaa.dk}
        \item Åbne driftsmøder hver tirsdag efter ${T^3}$
      \end{itemize}
    \end{itemize}
  \end{itemize}
\end{frame}


\begin{frame}
  \frametitle{Hvad er der sket siden sidst?}
  \begin{itemize}
  \item SPACEAPI interface (Tech talk soon...)
  \item Omflytning ...
  \item Indførsel af ${T^3}$ rundvisning :)
  \item Limbo er tømt
  \item Bridgeporten er "funktionel"
  \end{itemize}
\end{frame}

\begin{frame}
  \frametitle{Hvor kan jeg se hvad der sker?}
  \textbf{Følg med på}
  \begin{itemize}
  \item  \url{http://osaa.dk}
  \item  \url{https://groups.google.com/group/openspaceaarhus}
  \item  \url{https://twitter.com/openspaceaarhus} og \texttt{\#osaa}
  \item  \url{https://facebook.com/groups/openspaceaarhus/}
  \end{itemize}
\end{frame}

\begin{frame}
  \frametitle{Tech Talk Tuesday}
  \begin{itemize}
  \item Hvad - små tekniske foredrag på 15-30 min.
  \item Hvorfor - få et indblik i mange forskellige emner
  \item Hvornår - Hver 1. tirsdag i måneden kl. 19
  \item Hvem - Kom til os, hvis du har ideer til emner eller oplægsholdere
  \item Hvor - \url{http://t3.osaa.dk}
  \end{itemize}
\end{frame}

\begin{frame}
  \frametitle{Aftenens program}
  \begin{itemize}
    % \item Nyt fra OSAA - \emph{hvad sker der i lokalerne?}
  \item  Mark: \emph{OSAA} intro og nyheder (this is it)
  \item  (15 min) René Møller: JavaScript – sproget, alle hader eller the new hot shit?
  \item  (30 min) Boris Persson (oz2lpx): Introduktion til radioamatør hobbyen
  \item  30 min pause - networking, sodavand, snacks til rimelige priser.
  \item  (15 min) Carsten Agger: Versions- og release-styring med git og git flow [som vi bruger hos Magenta]
  \item  (15 min) Markus Wüstenberg: Videre med Github – Versionskontrol og derudover
  \item  (15 min) Mark Moore / Emil Madsen: Coding Pirates – \newline Et frivillig initiativ der introducerer børn til teknologi

  \end{itemize}
\end{frame}

% \begin{frame}
%   \frametitle{Kommende aktiviteter}
%   \begin{itemize}
%   \item OSAA bestyrelsesmøde: Trsdag d. 11/11 2014 kl. 19
%   \item Næste ${T^3}$: Tirsdag d. 2/12 2014 kl. 19.00
%   \end{itemize}
% \end{frame}


% \begin{frame}
%   \frametitle{Pause}
%   \begin{itemize}
%   \item Snak med dine mednørder
%   \item Køb snacks
%   \item Hvis du har spørgsmål/ideer til OSAA, så find et af bestyrelsesmedlemmerne
%   \end{itemize}
%   % \includegraphics[width=1.0\textwidth]{Neverending}
% \end{frame}




% \begin{frame}
%   \frametitle{Kommende arrangementer}
%   \begin{itemize}
%   \item Vi ses næste gang til:
%     \begin{itemize}
%     \item Hack Århus klubaften hver tirsdag kl. 17-23
%     \item Ubuntu Århus torsdage i lige uger kl. 17-19
%     \item OSAA bestyrelsesmøde torsdag d. 4/11 2010 kl. 19
%
%     \item Ubuntu release-arrangement lørdag d. 20. november
%     \item Næste ${T^3}$ tirsdag d. 7/12 2010 kl. 19.00
%     \end{itemize}
%   \item Følg med på \url{http://osaa.dk}
%   \end{itemize}
% \end{frame}

% \begin{frame}
%   \frametitle{Lidt om foreningen}
%   \begin{itemize}
%   \item
%			\item
%   \end{itemize}
% \end{frame}
% \begin{frame}
%   \frametitle{Oversigt}
%   \tableofcontents
% \end{frame}

\end{document}

